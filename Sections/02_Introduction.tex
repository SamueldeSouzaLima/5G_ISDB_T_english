Section {Introduction}
%======================================================
Providing broadband access to the regions that do not have adequate coverage and meet the growing demand for traffic where the frequency resources are not, are crucial challenges for fifth generation of mobile 5G. One of the scenarios proposed for the next generation of mobile telephony is to enable mainly wireless broadband Internet access for rural areas, the WRAN vignettes (Texte {Wireless regional Area Network}) cite {IEEE_802_22}. The IEEE 802.22 standard deals with this challenge by proposing the exploration of idle channels, making use of network s of cognitive radios (Texte {RC}) cite {Wran}. Couple
%=====================================================
Weliston most of the electromagnetic spectrum has already been licensed by the regulatory agency, several measurements campaigns show that the rates of occupation of the electromagnetic spectrum are between 5% and 15% cite {IndoorWS} cite {White_Space _capacity }. With the analog TV shutdown process, a significant part of the spectrum in the VHF (very High Frequency) and UHF (Textif {Ultra High frequency}) range is being unoccupied cite {White_Space_Capacity, AnalogSwitchOff}. Parts that before Centrico localization for users products, license holders use the spectrum, are now availble for drug users. These gaps in the frequency spectrum are referred to as texte {blanks} cite {WSBook}. Couple
%=====================================================
In this context, the use of cognitive radios operating in the idle gaps of the spectrum is a solution to meet the needs observed, by allowing users to use the unlicensed tracks, optimizing the resources availble cite { WSBook}. However it is necessary to Absence that they do not cause El in the users products, spectrum holders. Couple
%=====================================================
In cite {GFDM}, the authors present a modulation technique with non-orthogonal carrier languages suitable for RC, because it allows the fragmented use of the spectrum and the control of emission to the range. This modulation technique is known as GFDM (textity(generalized Frequency Division multiplexing) and can be considered a generalization of the OFDM (Texti(orthogonal Frequency Division multiplexing)) cite {ref_OFDM}. The main feature that puts the GFDM as a promising transmission pattern for 5G is its flexibility to cite {heirOFDM}. Another technique proposed to decrease the emission is that of the range and decrease the emission of spurious is the use of the technique F-OFDM (Texte {Filtered-OFDM}) cite {FOFDM_hwawei} being a strong candidate to be adopted in was scenarios if for 5g. Couple
%=====================================================
Given the possibility of joint operation between TV systems and mobile communication systems in the UHF range, this work aims to Get a proposal to identify, analyse and measure interoperability between systems based on waveforms Proposals for the next generation cell operating in company channels in the TV range, and the current standard of Brazilian broadcasting the ISDB-T (TEXTT {Integrated Services Digital terrestrial Broadcasting). Through the analysis of the results, we evaluated the coexistence of legacy systems with the proposed waveforms for fifth generation of mobile communications. Couple
%====================================================
This article is organized as follows: In section II is presented a brief description of the waveforms studied, in section III describes the as system for conducting the experiments was assembled. In section IV, the procedures for conducting the tests are submitted and in section V the results of the interoperability test and coexistence of the different technologies studied. Finally, in section VI, the conclusions are elucidated.  Couple

\section{Introdução}
%======================================================
Prover acesso de banda larga para as regiões que não possuem cobertura adequada e atender à crescente demanda de tráfego onde os recursos de frequência são limitados, são desafios cruciais para quinta geração de celular o 5G. Um dos cenários propostos para a próxima geração da telefonia celular é viabilizar principalmente o acesso à Internet de banda larga sem fio para áreas rurais, as chamadas WRAN (\textit{Wireless regional area network}) \cite{IEEE_802_22}. O padrão IEEE 802.22 trata deste desafio propondo a exploração de canais ociosos, fazendo o uso de redes de rádios cognitivos (\textit{RC})\cite{wran}. \par
%=====================================================
Embora a maior parte do espectro eletromagnético já tenho sido licenciado pela agência reguladora, várias campanhas de medições realizadas mostram que as taxas de ocupação do espectro eletromagnético estão entre 5\% e 15\% \cite{IndoorWS}\cite{White_Space_Capacity}. Com o processo de desligamento da TV analógica, uma parte significativa do espectro na faixa de VHF (\textit{Very High Frequency}) e UHF (\textit{Ultra High Frequency}) está sendo desocupada \cite{White_Space_Capacity,AnalogSwitchOff}. Partes que antes estavam reservadas para usuários primários, detentores da licença de uso do espectro, agora estão disponíveis para usuários secundários. Estas lacunas no espectro de frequência são denominados de \textit{White Spaces} \cite{WSBook}. \par
%=====================================================
Neste contexto o emprego de rádios cognitivos operando nas lacunas ociosas do espectro é uma solução para atender as necessidades observadas, ao permitir que usuários secundários utilizem as faixas não licenciadas, otimizando os recursos disponíveis \cite{WSBook}. Entretanto é preciso garantir que os mesmos não causem interferências nos usuários primários, detentores do espectro. \par
%=====================================================
Em \cite{gfdm}, os autores apresentam uma técnica de modulação com múltiplas portadoras não ortogonais adequada para RC, pois permite o uso fragmentado do espectro e o controle da emissão fora da faixa. Esta técnica de modulação é conhecida como GFDM (\textit{Generalized Frequency Division Multiplexing}) e pode ser considerada uma generalização do OFDM (\textit{Orthogonal Frequency Division Multiplexing}) \cite{ref_OFDM}. A principal característica que coloca o GFDM como um padrão de transmissão promissor para o 5G é sua flexibilidade \cite{heirOFDM}. Uma outra técnica proposta para diminuir a emissão fora da faixa e diminuir a emissão de espúrios é a utilização da técnica F-OFDM (\textit{Filtered-OFDM}) \cite{fOFDM_hwawei} sendo uma forte candidata à ser adotada em alguns cenários previstos para o 5G. \par
%=====================================================
Dada a possibilidade de operação conjunta entre sistemas de TV e sistemas de comunicação móvel na faixa de UHF, este trabalho tem como objetivo apresentar uma proposta de para identificar, analisar e medir a interoperabilidade entre sistemas baseados em formas de ondas propostas para a próxima geração celular operando em canais ociosos na faixa de TV, e o atual padrão de radiodifusão brasileiro o ISDB-T (\textit{Integrated Services Digital Broadcasting Terrestrial}). Através da análise dos resultados, avaliou-se a coexistência de sistemas legados com as formas de onda propostas para quinta geração das comunicações móveis. \par
%====================================================
O presente artigo está organizado da seguinte maneira: Na Secção II é apresentado uma breve descrição das formas de ondas estudadas, na secção III descreve-se o como sistema para realização dos experimentos foi montado. Na secção IV são descritos os procedimentos para realização dos ensaios e na secção V os resultados dos teste de interoperabilidade e coexistência das diferentes tecnologias estudadas. Por fim na secção VI, as conclusões são elucidadas.\par
