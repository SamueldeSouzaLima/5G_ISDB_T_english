Section {Conclusions}
%========================================================
The objective of this paper was to purpose a method of test procedure and evaluate the coexistence of different technologies ...
that are candidates for the next generation of mobile communications,  operating in an opportunistic way in the spectrum in coexistence with the Brazilian standard of digital TV ISDB-T. Couple

5G WRAN 
%=========================================================
Though a small number of ISDB-T receivers were used, it was/ is affirmed that there are considerable variations between the reception thresholds according to the receptor model.
The tests presented in this article demonstrate that transmission techniques proposed for 5G as GFDM janellated and F-OFDM have advantages when considering the emission for the range and fragmented use of the spectrum. From these values one can estimate what would be the limitations in the characteristics of operation or installation of systems transmitting in secondary character. Couple

%=======================================================
This work aims to assist in the survey of protection relations, to determine the emission masks that allow 5G systems to coexist with the existing products systems. Couple
